\chapter{Introduction}\label{chap.intro}
\minitoc
\section{Preamble}
\subsection{Industry 4.0}
Industry is a highly mechanized and automatized piece of an economy that produces goods. Since the beginning of industrialization, technological leaps have led to paradigm shifts, now called "industrial revolutions": from mechanization, electrification, and later, digitalization (the so-called 3rd industrial revolution). Based on the advanced digitalization within factories, the combination of Internet technologies and future-oriented technologies in the field of "smart" things (machines and products) seems to result in a new fundamental paradigm shift in industrial production. Emerging from this future expectation, the term "Industry 4.0" was established for an expected "4th industrial revolution" \cite{lasi2014industry}.


\subsection{Internet-of-Things in Industry}
% Context background
To build the emerging environment of Industry 4.0, disruptive technologies are required to handle autonomous communications between all industrial embedded computers throughout the factory and the Internet. Such technologies offer the potential to transform the industry along the entire production chain and stimulate productivity and overall economic growth \cite{espinoza2020estimating}. These technologies include cloud computing, big data, and specially a new generation of \gls{iot} devices fused with \gls{cps}, safety-security, augmented reality, \gls{ml}, and hardware accelerators \cite{alcacer2019scanning}.

\subsection{Artificial Intelligence in Internet-of-Things}
% Particular background
The continuous evolution of \gls{ai} algorithms and \gls{iot} devices has not only made \gls{ai} the major workload running on these embedded devices, but \gls{ai} is becoming the main approach for industrial solutions, especially in the rise of Industry 4.0 \cite{alcacer2019scanning}. There is a clear motivation to run \gls{ai}/\gls{ml} algorithms on \gls{iot} devices because of \cite{loh20201}: (1) feasibility of mission-critical with real-time processing; (2) privacy and security of data; (3) offline operation capability; and (4) robustness for stressed communication. Hence, the term of \gls{iot} has also been redefined as AI of Things (AIoT) to emphasize the impact of \gls{ai}/\gls{ml} on this technology \cite{zhang2020empowering}.

\subsection{Error Tolerance in Machine Learning Algorithms}
An algorithm can be regarded as error-tolerant or error-resilient when it provides a result with the required accuracy while utilizing processing components with a certain degree of inaccuracy. There are several reasons why an algorithm/application is tolerant of errors as discussed in \cite{chippa2013analysis}. These include noisy or redundant data of the algorithm, approximate or probabilistic computations within the algorithm, and a range of acceptable outcomes. This is the case of \gls{ml} models for \gls{ai} applications.

% Problem to solve
\section{Problem Statement}
The problem lies in the fact that state-of-the-art \gls{ai}/\gls{ml} models, particularly \glspl{cnn} and \glspl{ann}, are highly computational and data intensive. This brings significant challenges across the spectrum of computing hardware, specially in the scope of embedded systems \cite{venkataramani2016efficient}. The most deployed models and also the most computationally and energy expensive are for computer vision using \glspl{cnn}. Compared to the conventional image processing methods, the accuracy of \gls{cnn} has improved significantly that by 2015, a human can no longer beat a computer in image classification \cite{loh20201}. The early development of \glspl{cnn} before 2016 mainly focused on accuracy improvement without considering computational costs. While accuracy of deep \gls{cnn} for image classification improved 24\% between 2012 and 2016, the demand on hardware resources increased more than $10\times$. Starting from 2017, significant attention was paid to improve hardware efficiency in terms of compute power, memory bandwidth, and power consumption, while maintaining accuracy at a similar level to human perception \cite{venkataramani2016efficient}.

% Consequences of the problem
Consequently, the recent breakthroughs in \gls{ai}/\gls{ml} applications have brought significant advancements in neural network processors \cite{jouppi2017datacenter}. These rapid evolution, however, came at the cost of an important demand for computational power. Hence, to bring the inference speed to an acceptable level, \gls{asic} with \gls{npu} are becoming ubiquitous in both embedded and general purpose computing. \glspl{npu} perform several tera operations per second in a confined area, as a consequence, they become subject to elevated on-chip power densities that rapidly result in excessive on-chip temperatures during operation \cite{amrouch2020npu}. This outcome is expected on parallel computing techniques, yet unsustainable for resource-constrained devices.
Therefore, radical changes to conventional computing are required in order to sustain and improve performance while satisfying energy and temperature constraints \cite{gillani2020exploiting}.

In the state-of-the-art research, we find plenty of hardware architectures for \gls{cnn} accelerators implemented in \gls{fpga}. Most of the research implements fixed-point quantization, and very limited research focuses on \gls{fp}. Moreover, to the best of our knowledge, there is no research work exploring custom or conventional \gls{fp} inference for low-power embedded systems.

%%%%%%%%%%%%%%%%%%%%%%%%%%
\section{Research Objective}
The main objective for this doctoral research is investigating hardware design methodologies for low-power \gls{fp} neural network accelerators based on approximate computing with high quality of inference in the scope of embedded systems.

%%%%%%%%%%%%%%
\section{Working Hypothesis}
% Alternatives and possible solutions for the problem
To overcome the problem, based on the error resilience of \gls{ml} algorithms, an evident solution is approximate computing. This computing paradigm has been used in a wide range of applications to increase hardware efficiency \cite{han2013approximate}. For neural network applications, two main approximation strategies are used, namely network compression and classical approximate computing \cite{bouvier2019spiking}.

\subsection{Network Compression and Quantization}
Researchers focusing on embedded applications started lowering the precision of weights and activation maps to shrink the memory footprint of the large number of parameters representing \glspl{ann}, a method known as network quantization. In this manner, reduced bit precision causes a small accuracy loss \cite{courbariaux2015binaryconnect, han2015deep, hubara2017quantized, rastegari2016xnor}. In addition to quantization, network pruning reduces the model size by removing structural portions of the parameters and its associated computations \cite{lecun1989optimal,hassibi1992second}. This method has been identified as an effective technique to improve the efficiency of \gls{cnn} for applications with limited computational budget \cite{molchanov2016pruning,li2016pruning, liu2018rethinking}. These techniques leverage the intrinsic error-tolerance of neural networks, as well as their ability to recover from accuracy degradation while training.

\subsection{Approximate Computing}
%Geneal background
Approximate computing is an emerging design paradigm that is able to tradeoff computation quality (e.g., accuracy) and computational efficiency (e.g., in run-time, chip-area, and/or energy) by exploiting the error resilience properties of algorithms/applications \cite{gillani2020exploiting, zhang2015approxann}. Data redundancy of neural networks incorporate a certain degree of resilience against random external and internal perturbations, for instance, noisy inputs and random hardware errors. This property can be exploited in a cross-layer resilience approach \cite{carter2010design}: by leveraging error tolerance at algorithmic-level, it can be allowed a certain degree of inaccuracies at the computing-level. This approach consists of designing processing elements that approximate their computation by employing cleverly modified algorithmic logic units \cite{han2013approximate}.

Approximate computing techniques allow substantial enhancement in processing efficiency with moderated accuracy degradation. Some research papers have shown the feasibility of applying approximate computing to the inference stage of neural networks \cite{lotrivc2012applicability, han2013approximate, du2014leveraging, mrazek2016design, sarwar2016multiplier, zervakis2021approximate}. Such techniques usually demonstrated small inference accuracy degradation, but significant enhancement in computational performance, chip-area, and energy consumption. Hence, by taking advantage of the intrinsic error-tolerance of neural networks, approximate computing is positioned as a promising approach for inference on resource-limited devices. Nonetheless, the complex state-of-the-art of \gls{fp} \gls{cnn} inference has not been sufficiently explored with approximate computing techniques.

\section{Motivation}\label{chap1.motivation}
The use of \gls{ai}/\gls{ml} is entering a new era with ubiquitous embedded connected devices. This transformation requires design techniques that reconcile accurate results with cost-effective system architectures. However, state-of-the-art \gls{ai}/\gls{ml} algorithms represent high computational and energy costs. This compromises the sustainability of the progressive expansion towards massive ubiquitous \gls{ai} devices. Therefore, hardware efficiency is one of the major goals to innovate compute engines. This section presents the motivations to investigate design methodologies for low-power hardware acceleration for \gls{sbs} and \glspl{cnn}.

\begin{itemize}

\item \textbf{Spike-by-Spike Neural Networks}. \glspl{snn} offer advantageous robustness and the potential to achieve a power efficiency closer to that of the human brain. \glspl{snn} operate reliably using stochastic elements that are inherently non-reliable mechanisms \cite{mcdonnell2011benefits}. This provides superior resistance against adversary attacks
\cite{ernst2007efficient, Dapello2020.06.16.154542}. Beside robustness, \glspl{snn} have further advantages like the possibility of a more efficient asynchronous parallelization and higher energy efficiency than conventional \glspl{ann}.

The Spike-by-Spike neural network is a remarkable model for its reduced complexity, which is on the less realistic side of the \gls{snn} scale of biological realism \cite{rotermund2019Backpropagation,ernst2007efficient}. Consequently, the hardware complexity of \gls{sbs} network implementations is greatly reduced \cite{rotermund2018massively}. In spite of this, \gls{sbs} still uses stochastic spikes as a means of transmitting information between populations of neurons and thus retains the advantageous robustness of \glspl{snn}. A significant research effort has been performed in \gls{snn} accelerators, see e.g. \cite{roy2019towards,bouvier2019spiking,
	young2019review,TrueNorth_Trans15,Spinnaker_Trans13,davies2018loihi}.

However, hardware accelerators that focus on \gls{sbs} have only been partially investigated so far \cite{rotermund2018massively}. Enhanced \gls{sbs} accelerators will have a double impact. From application point of view, they will contribute to the deployment of robust neural networks in small embedded systems \cite{nevarez2020accelerator}; from a scientific point of view, they will facilitate fundamental research for neuroscience \cite{ernst2007efficient,rotermund2019recurrentsbs, dayan2001theoretical}.

%%%%%%%%%%%%%%%%%%%%%%%%%%%%%%%%%%%%%%%%%%%%%%%%%%%%%%%%%%%%%%%%%%%%%%
\item \textbf{Convolutional Neural Networks}. \glspl{cnn} represent the essential building blocks in 2D pattern analytics. Sensor-based applications such as mechanical fault diagnosis \cite{li2019sensor,dong2018rolling}, structural health monitoring \cite{nagayama2007structural}, \gls{har} \cite{wang2019deep}, hazardous gas detection \cite{kim2017hazardous} have been powered by \gls{cnn} models in industry and academia. \gls{cnn} models provide advantages such as local dependency, scale invariance, and noise resilience in analytics \cite{du2014leveraging}. However, these models are computationally intensive and power-hungry. This is particularly challenging for low-power embedded applications in the field of \gls{iot}. As a result, numerous commercial \gls{asic} and \gls{fpga} accelerators have been proposed, targeting both \gls{hpc} for data-centers and embedded systems applications.

However, most accelerators have been implemented to target mid- to high-range \glspl{fpga} for computationally intensive \gls{cnn} models such as AlexNet, VGG-16, and ResNet-18. The main drawbacks of these implementations are power supply demands, physical dimensions, heat sink requirements, air cooling, and a resulting high price. In some cases, these implementations are not feasible for ubiquitous low-power/resource-constrained applications. Furthermore, reducing the compute hardware with aggressive quantization such as binary \cite{courbariaux2015binaryconnect}, ternary \cite{lin2015neural}, and mixed precision (2-bit activations and ternary weights) \cite{colangelo2018exploration} typically incur significant accuracy degradation for very low precisions, especially for complex problems\cite{faraone2019addnet}.

\end{itemize}

\section{Main Contribution}
This thesis contributes to hardware design methodologies for \gls{fp} neural network accelerators based on approximate computing in the scope of low-power embedded systems. The contributions for \gls{sbs} and \gls{cnn} hardware accelerators are listed below.

\subsection{Spike-by-Spike Neural Networks}
\begin{enumerate}
	\item We develop a hardware component for \gls{fp} vector dot-product approximation. This design increase the performance of computation by performing element-wise multiplication with a quality configurable design based on bit truncation and denormalized accumulation.
	\item We address a design exploration with the proposed dot-product approximation using synaptic weight vectors with custom \gls{fp} and logarithmic representation. We evaluate inference latency, accuracy degradation, resource utilization and power dissipation. Experimental results demonstrate $20.5\times$ latency enhancement versus embedded \gls{cpu} (ARM Cortex-A9 at $666$MHz), and less than $0.5\%$ of accuracy degradation on a handwritten digit recognition task (MNIST).
	\item We propose a noise tolerance plot as quality monitor, which serves as an intuitive visual model to provide insights into the accuracy degradation of \gls{sbs} networks under approximate processing effects.
	\item Our proposed design for \gls{fp} dot-product approximation is adaptable as a building block for other error resilient applications (e.g., image/video processing).
\end{enumerate}


\subsection{Convolutional Neural Networks}
\begin{enumerate}
	\item
	
	We present the \gls{hf6} quantization and its dedicated hardware design. We propose an optimized hardware \gls{mac} by reducing the mantissa multiplication to a multiplexer-adder operation. We exploit the intrinsic error tolerance of \gls{ann} to further reduce the hardware design with approximation. To preserve model accuracy, we present a quantization-aware training method, which in some cases improves accuracy.
	
	\item We develop a custom hardware/software co-design framework for sensor analytics applications on low-power embedded \glspl{fpga}. This architecture integrates TensorFlow Lite.
	\item We present a customizable tensor processor as a dedicated hardware for \gls{hf6}. This design computes \emph{Conv2D} tensor operations employing a pipelined vector dot-product with parametrized on-chip memory utilization.
	\item We demonstrate the potential of our approach with a \gls{cnn}-regression model for anomaly localization in structural health monitoring based on acoustic emissions. We address a hardware design exploration evaluating accuracy, compute performance, hardware resource utilization, and energy consumption.
\end{enumerate}

\section{Publications}
The outcome of the research of this dissertation, including the collaborative works with our research partners is a list of publications including \cite{nevarez2020accelerator, nevarez2021accelerating, yu2020taac}. In the following, a complete list of the related publications are itemized.

\begin{enumerate}
	
	\subsubsection*{Journal Articles}
	
	\item \textbf{Yarib Nevarez}, David Rotermund, Klaus R Pawelzik, and Alberto Garcia-Ortiz, "Accelerating Spike-by-Spike Neural Networks on FPGA With Hybrid Custom Floating-Point and Logarithmic Dot-Product Approximation," 
	\newblock IEEE Access, vol. 9, pp. 80603--80620, May 2021, doi: 10.1109/ACCESS.2021.3085216.  
	
	\item \textbf{Yarib Nevarez}, Andreas Beering, Amir Najafi, Ardalan Najafi, Wanli Yu, Yizhi Chen, Karl-Ludwig Krieger, and Alberto Garcia-Ortiz, "CNN Sensor Analytics With Hybrid-Float6 Quantization on Low-Power Embedded FPGAs," 
	\newblock IEEE Access, vol. 11, pp. 4852--4868, January 2023, doi: 10.1109/ACCESS.2023.3235866.

	
	\subsubsection*{Conference Proceedings}
	
	\item \textbf{Yarib Nevarez}, Alberto Garcia-Ortiz, David Rotermund, and Klaus R Pawelzik, "Accelerator framework of spike-by-spike neural networks for inference and incremental learning in embedded systems,"
	\newblock 2020 9th International Conference on Modern Circuits and Systems Technologies (MOCAST), Bremen, 2020, pp. 1--5, doi: 10.1109/MOCAST49295.2020.9200288.
	
	\item Wanli Yu, Ardalan Najafi, \textbf{Yarib Nevarez}, Yanqiu Huang and Alberto Garcia-Ortiz, "TAAC: Task Allocation Meets Approximate Computing for Internet of Things," 
	\newblock 2020 IEEE International Symposium on Circuits and Systems (ISCAS), Sevilla, 2020, pp. 1-5, doi: 10.1109/ISCAS45731.2020.9180895.
	
	\item Amir Najafi, Ardalan Najafi, \textbf{Yarib Nevarez} and Alberto Garcia-Ortiz, "Learning-Based On-Chip Parallel Interconnect Delay Estimation," 
	\newblock 2022 11th International Conference on Modern Circuits and Systems Technologies (MOCAST), Bremen, 2022, pp. 1--5, doi: 10.1109/MOCAST49295.2020.9200288.
	
	\item Yizhi Chen, \textbf{Yarib Nevarez}, Zhonghai Lu, and Alberto Garcia-Ortiz, "Accelerating Non-Negative Matrix Factorization on Embedded FPGA with Hybrid Logarithmic Dot-Product Approximation," 
	\newblock 2022 IEEE 15th International Symposium on Embedded Multicore/Many-core Systems-on-Chip (MCSoC), Malaysia, 2022, pp. 239--246, doi: 10.1109/MCSoC57363.2022.00070.
	
\end{enumerate}

\section{Dissertation Outline}

This dissertation is organized in three main parts: an introduction, where
the state of the art and related background are stated; a central core, where the proposed design methodologies and validation are presented; and a final part with the conclusion. More precisely:

\begin{enumerate}[I]
	\item \textbf{Introduction}: Chapter~\ref{chap.background} introduces the background related to \gls{sbs}, \gls{cnn}, and \gls{fp} number representation.
	\item \textbf{Core}: the proposed hardware design methodologies for \gls{sbs} and \gls{cnn} accelerators are presented in Chapter~\ref{chap.sbs} and Chapter~\ref{chap.cnn}, respectively.
	\item \textbf{Conclusions}: the final conclusions are presented in Chapter~\ref{chap.conclusion}.
\end{enumerate}