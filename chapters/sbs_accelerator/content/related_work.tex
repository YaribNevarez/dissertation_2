\section{Related Work}
\label{sec:related_work}
%%%%%%%%%%%%%%%%%%%%%%%%%%%%%%%%%%%%%%%%%%%%%%%%%%%%%%
For efficient neural network computation, two main optimization strategies are used, namely network compression and classical approximate computing~\cite{bouvier2019spiking}.

\subsection{Network Compression}
Researchers focusing on embedded applications started lowering the precision of weights and activation maps to shrink the memory footprint of the large number of parameters representing \glspl{ann}, a method known as network compression or quantization. This practice takes advantage of the intrinsic error-tolerance of neural networks, as well as their ability to compensate for approximation while training. In this way, reduced bit precision causes a small accuracy loss~\cite{courbariaux2015binaryconnect, han2015deep, hubara2017quantized, rastegari2016xnor}.

In hardware development, \gls{wq} has shown up to $2\times$ improvement in energy consumption with an accuracy degradation of less than $1\%$ \cite{moons20160, whatmough201714}. Some advanced quantization methods yield to \glspl{bnn} allowing the use of \glspl{xnor} instead of the conventional costly \glspl{mac}~\cite{rastegari2016xnor}. In~\cite{sun2018xnor}, Sun et al. report an accuracy of $98.43\%$ on handwritten digit classification (MNIST) with a simple \gls{bnn}. Hence, quantization is a powerful tool for improving the energy efficiency and memory requirements of \gls{ann} accelerators, with limited accuracy degradation.

In addition to quantization, network pruning reduces the model size by removing structural portions of the parameters and its associated computations~\cite{lecun1989optimal,hassibi1992second}. This method has been identified as an effective technique to improve the efficiency of \gls{dnn} for applications with limited computational budget~\cite{molchanov2016pruning,li2016pruning, liu2018rethinking}.

These methods can be used for \glspl{snn} as well. In~\cite{rathi2018stdp}, Rathi et al. report up to $3.1\times$ improvement in energy consumption with an accuracy loss of around $3\%$. Weight quantization allows the designer to realize a trade-off between the accuracy of the \gls{snn} application and efficiency of resources. Approximate computing can also be applied at the neuron level, where irrelevant units are deactivated to reduce the computation cost of the \glspl{snn}~\cite{sen2017approximate}. This computation skipping can be applied randomly on synapses, training \glspl{ann} with stochastic synapses improves generalization, resulting in a better accuracy~\cite{srivastava2014dropout, wan2013regularization}. Such methods are compatible with \glspl{snn} and have been tested both during training~\cite{neftci2016stochastic, srinivasan2016magnetic} and operation \cite{buesing2011neural}, and even to define the connectivity between layers \cite{bellec2017deep, chen20184096}. Implementations of spiking neuromorphic systems in \gls{fpga}~\cite{sheik2016synaptic} and hardware~\cite{jerry2017ultra} demonstrated that synaptic stochasticity allows to increase the final accuracy of the networks while reducing memory footprint.

Quantization is therefore a powerful technique to improve energy efficiency and memory requirements of \gls{ann} and \gls{snn} accelerators, with small accuracy degradation. However, this approach requires quantization-aware training methods that, in some cases, are problematic or even inaccessible, particularly in emerging deep \gls{snn} algorithms~\cite{zhang2018survey}.

\subsection{Classical Approximate Computing}
Approximate computing has been used in a wide range of applications to increase the computational efficiency in hardware~\cite{han2013approximate}. This approach consists of designing processing elements that approximate their computation by employing modified algorithmic logic units~\cite{han2013approximate}. In~\cite{kim2013energy}, Kim et al. have shown \glspl{snn} using carry skip adders achieving $2.4\times$ latency enhancement and $43\%$ more energy efficiency, with an accuracy degradation of 0.97\% on a handwritten digit classification task (MNIST). Therefore, approximate computing provides important enhancement in energy efficiency and processing speed.

However, as the complexity of the dataset increases, as well as the depth of the network topology, such as ResNet~\cite{he2016deep} on ImageNet~\cite{russakovsky2015imagenet}, the accuracy degradation becomes more important and may not be negligible anymore~\cite{rastegari2016xnor}, especially for critical applications such as autonomous driving. Therefore, it is not certain that network compression techniques and approximate computing are suitable for all applications.

\subsection{Spike-by-Spike Neural Networks Accelerators}
Rotermund et al. demonstrated the feasibility of a neuromorphic \gls{sbs} \gls{ip_sbs} on a Xilinx Virtex 6 \gls{fpga}~\cite{rotermund2018massively}. It provides a massively parallel architecture, optimized to reduce memory access and suitable for \gls{asic} implementations. Nonetheless, this design is considerably resource-demanding if implemented as a full \gls{sbs} network in today's embedded technology.
%%%%%%%%%%%%%%%%%%%%%%%%%%%%%%%%%%%%%%%%%%%%%%%%%%%%%%